\documentclass[10.5pt,a4]{article}

\usepackage{verbatim}
\usepackage{listings}
\lstdefinelanguage{modelica}
{
  keywords={
    algorithm,and,annotation,as,assert,block,break,case,class,connect,connector,
    constant,constrainedby,der,discrete,each,else,elseif,elsewhen,encapsulated,
    end,enumeration,equality,equation,expandable,extends,external,failure,final,
    flow,for,function,guard,if,import,in,initial,inner,input,List,local,loop,
    match,matchcontinue,model,not,operator,Option,or,outer,output,package,parameter,
    partial,protected,public,record,redeclare,replaceable,return,stream,
    subtypeof,then,Tuple,type,uniontype,when,while,within},
  % Do not make true,false keywords because fn(true,x, false ) shows up as fn(true,x, *false*)
  sensitive=true,
  comment=[l]{//},
  morecomment=[s]{/*}{*/},
  morecomment=[s]{"}{"},
  alsodigit={.,-},
}

\usepackage[utf8]{inputenc}
\usepackage{url}
\lstset{breaklines=true,language=modelica,basicstyle=\small}
\usepackage{graphicx}
\usepackage{color}

\lstset{keywordstyle = \bf}

\newcommand{\code}[1]{\texttt{\hyphenchar%
\font45%
\sloppy%
\fontdimen2\font=0.4em%
\fontdimen3\font=0.2em%
\fontdimen4\font=0.1em%
\fontdimen7\font=0.1em%
 #1}}

% \pagestyle{empty}
\usepackage{geometry}
\geometry{left=27mm, right=27mm, top=16mm, bottom=27mm, noheadfoot}
\usepackage[a4paper,colorlinks,linkcolor=blue,citecolor=black,urlcolor=blue]{hyperref}
\usepackage[all]{hypcap}

\begin{document}
\appendix

\begin{comment}
\section{Dummy}
...

\section{A lot of stuff is missing here compared to the book...}
\subsection{Interface to the Standard MetaModelica Package}
\lstinputlisting{MetaModelicaBuiltin.mo}
\end{comment}

\addtocounter{section}{2}

\section{Complete Small Language Specifications}
This appendix contains several small language specifications, both interpretive and translational.

\subsection{The Complete Interpretive Semantics for PAM}
\label{listings:pam}
\lstinputlisting{../../08_pam/Pam.mo}

\subsection{Complete PAMDECL Interpretive Specification}
The complete PAMDECL interpretive specifiation is presented below. The packages/files are shown starting with the more interesting semantics packages, ending with the lexer and parsing files.

\label{listings:pamdecl}

\subsubsection{PAMDECL Main Package}
The main package implements the read-eval-print function as the function evalprog, which accepts the initial environment initial containing only true and false exported from package Eval.
The main package of the PamDecl evaluator calls ScanParse to read and parse text from the standard input, and Eval to evaluate and print the results.
\lstinputlisting{../../09_pamdecl/Main.mo}

\subsubsection{PAMDECL ScanParse}
The ScanParse package contains only one function scanparse, which is an external function implemented in C to scan and parse text written in the PamDecl language.
\lstinputlisting{../../09_pamdecl/ScanParse.mo}

\subsubsection{PAMDECL Absyn}
The Absyn package specifies the abstract syntax used by the rest of the specification, i.e., the other packages.
\lstinputlisting{../../09_pamdecl/Absyn.mo}

\subsubsection{PAMDECL Env}
The Env package contains functions and types for describing and handling environments in a declarative way, including building environments and performing lookup on environments.
\lstinputlisting{../../09_pamdecl/Env.mo}

\subsubsection{PAMDECL Eval}
The Eval package contains the interpretive semantic cases, collected in functions, for all the constructs in PAMDECL, including expressions, statements, and declarations.
\lstinputlisting{../../09_pamdecl/Eval.mo}

\lstset{language=}
\subsubsection{PAMDECL lexer.l}
\lstinputlisting{../../09_pamdecl/lexer.l}

\subsubsection{PAMDECL parser.y}
\lstinputlisting{../../09_pamdecl/parser.y}
\lstset{language=modelica}

\subsubsection{PAMDECL Makefile}
\lstinputlisting{../../09_pamdecl/Makefile}

\subsection{The Complete PAM Translational Specification}
The following files are needed for building the PAM translator: \code{Absyn.mo}, \code{Trans.mo}, \code{Mcode.mo}, \code{Emit.mo}, \code{lexer.l}, \code{parser.y}, \code{Main.mo}, \code{Parse.mo} and \code{Makefile}.
\label{listings:pamtrans}

\subsubsection{PAMTRANS Absyn.mo}
\lstinputlisting{../../10_pamtrans/Absyn.mo}

\subsubsection{PAMTRANS Trans.mo}
\lstinputlisting{../../10_pamtrans/Trans.mo}

\subsubsection{PAMTRANS Mcode.mo}
\lstinputlisting{../../10_pamtrans/Mcode.mo}

\subsubsection{PAMTRANS Emit.tpl}
\lstinputlisting{../../10_pamtrans/Emit.tpl}

\subsubsection{PAMTRANS EmitTV.mo}
\lstinputlisting{../../10_pamtrans/EmitTV.mo}

\subsubsection{PAMTRANS Main.mo}
\lstinputlisting{../../10_pamtrans/Main.mo}

\subsubsection{PAMTRANS Parse.mo}
\lstinputlisting{../../10_pamtrans/Parse.mo}

\lstset{language=}
\subsubsection{PAMTRANS lexer.l}
\lstinputlisting{../../10_pamtrans/lexer.l}

\subsubsection{PAMTRANS parser.y}
\lstinputlisting{../../10_pamtrans/parser.y}

\subsubsection{PAMTRANS Makefile}
\lstinputlisting{../../10_pamtrans/Makefile}
\lstset{language=modelica}

\begin{comment}
\section{Petrol}

Petrol is a Pascal-like language. The complete implementation is presented below.
\label{listings:petrol}

\subsection{Petrol Abstract.mo}
\lstinputlisting{../../11_petrol/Absyn.mo}

\subsection{Petrol FCode.mo}
\lstinputlisting{../../11_petrol/FCode.mo}

\subsection{Petrol FCEmit.mo}
\lstinputlisting{../../11_petrol/FCEmit.mo}

\subsection{Petrol Flatten.mo}
\lstinputlisting{../../11_petrol/Flatten.mo}

\subsection{Petrol Static.mo}
\lstinputlisting{../../11_petrol/Static.mo}

\subsection{Petrol TCode.mo}
\lstinputlisting{../../11_petrol/TCode.mo}

\subsection{Petrol Types.mo}
\lstinputlisting{../../11_petrol/Types.mo}

\subsection{Petrol Main.mo}
\lstinputlisting{../../11_petrol/Main.mo}

\subsection{Petrol Parse.mo}
\lstinputlisting{../../11_petrol/Parse.mo}

\lstset{language=}
\subsection{Petrol lexerPetrol.c}
\lstinputlisting{../../11_petrol/lexerPetrol.c}

\subsection{Petrol lexerPetrol.h}
\lstinputlisting{../../11_petrol/lexerPetrol.h}

\subsection{Petrol parser.y}
\lstinputlisting{../../11_petrol/parser.y}

\subsection{Petrol Makefile}
\lstinputlisting{../../11_petrol/Makefile}
\lstset{language=modelica}
\end{comment}

\section{Exercises}

\subsection{Exercises - Introduction and Interpretive Semantics}

\subsubsection{Exercise 01 Experiment – Types, Functions, Constants, Printing Values}
Additional example exercise that implements the pam language with declarations.

To run the calculator type \code{run} when building the project (or \code{make run} from a terminal).

If you need to edit the input of the calculator edit the file called \code{program.txt}.



The files where you should add your code are Functions.mo and Main.mo, also shown below:
\lstinputlisting{01/Functions.mo}
\lstinputlisting{01/Main.mo}


\subsubsection{Exercise 02a Exp1 –  Adding New Features to a Small Language}
Additional example exercise that implements the pam language with declarations.

To run the calculator type \code{run} when building the project (or \code{make run} from a terminal).

If you need to edit the input of the calculator edit the file called \code{program.txt}.


\lstinputlisting{02a/Exp1.mo}

\subsubsection{Exercise 02b Exp2 –  Adding New Features to a Small Language}
Additional example exercise that implements the pam language with declarations.

To run the calculator type \code{run} when building the project (or \code{make run} from a terminal).

If you need to edit the input of the calculator edit the file called \code{program.txt}.


\lstinputlisting{02b/Exp2.mo}

\subsubsection{Exercise 03 Symbolic Derivative –  Differentiating an Expression Using Symbolic Manipulation}
Additional example exercise that implements the pam language with declarations.

To run the calculator type \code{run} when building the project (or \code{make run} from a terminal).

If you need to edit the input of the calculator edit the file called \code{program.txt}.


\lstinputlisting{03/SymbolicDerivative.mo}

\subsubsection{Exercise 04 Assignment – Printing AST and Environments}
Additional example exercise that implements the pam language with declarations.

To run the calculator type \code{run} when building the project (or \code{make run} from a terminal).

If you need to edit the input of the calculator edit the file called \code{program.txt}.


\lstinputlisting{04/Assignment.mo}

\subsubsection{Exercise 05a AssignTwoType – Adding a New Type to a Language}
Additional example exercise that implements the pam language with declarations.

To run the calculator type \code{run} when building the project (or \code{make run} from a terminal).

If you need to edit the input of the calculator edit the file called \code{program.txt}.


\lstinputlisting{05a/AssignTwoType.mo}

\subsubsection{Exercise 05b ModAssignTwoType – Modularized Specification}
Additional example exercise that implements the pam language with declarations.

To run the calculator type \code{run} when building the project (or \code{make run} from a terminal).

If you need to edit the input of the calculator edit the file called \code{program.txt}.



\subsection{Exercises – Translational Semantics}
\subsubsection{Exercise 10 PAMTRANS – Small Translational Semantics}
Additional example exercise that implements the pam language with declarations.

To run the calculator type \code{run} when building the project (or \code{make run} from a terminal).

If you need to edit the input of the calculator edit the file called \code{program.txt}.



\subsubsection{Exercise 11 Petrol – Large Translational Semantics}
Additional example exercise that implements the pam language with declarations.

To run the calculator type \code{run} when building the project (or \code{make run} from a terminal).

If you need to edit the input of the calculator edit the file called \code{program.txt}.



\subsection{Exercises – Advanced}
\subsubsection{Exercise 06 Advanced – Polymorphic Types and Higher Order Functions}
Additional example exercise that implements the pam language with declarations.

To run the calculator type \code{run} when building the project (or \code{make run} from a terminal).

If you need to edit the input of the calculator edit the file called \code{program.txt}.


\lstset{language=modelica}

\section{Solutions to Exercises}

\subsection{Solution 01 Experiment – Types, Functions, Constants, Printing Values}
\lstinputlisting{01/SOLUTION.mo}

\subsection{Solution 02a Exp1 –  Adding New Features to a Small Language}
\lstinputlisting{02a/SOLUTION.txt}

\subsection{Solution 02b Exp2 –  Adding New Features to a Small Language}
Analogous to 02a Exp1.

\subsection{Solution 03 Symbolic Derivative –  Differentating an Expression Using Symbolic Manipulation}
\lstinputlisting{03/SOLUTION.mo}

\subsection{Solution 04 Assignment – Printing AST and Environments}
\lstinputlisting{04/SOLUTION.mo}

\subsection{Solution 05a AssignTwoType – Adding a New Type to a Language}
\lstinputlisting{05a/SOLUTION.mo}

\subsection{Solution 05b ModAssignTwoType – Adding a New Type to a Language}
Analogous to 05a AssignTwoType.

\subsection{Solution 06 Advanced – Polymorphic Types and Higher Order Functions}
\lstinputlisting{06/SOLUTION.mo}

\subsection{Solution 08 pam – A small Language}
The solution is available in Appendix~\ref{listings:pam}.

\subsection{Solution 09 pamdecl – Pam with Declarations}
The solution is available in Appendix~\ref{listings:pamdecl}.

\subsection{Solution 10 pamtrans – Small Translational Semantics}
The solution is available in Appendix~\ref{listings:pamtrans}.

%\subsection{Solution 11 petrol – Large Translational Semantics}
%The solution is available in Appendix~\ref{listings:petrol}.

\end{document}
